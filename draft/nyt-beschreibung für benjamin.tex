%% LyX 2.0.7 created this file.  For more info, see http://www.lyx.org/.
%% Do not edit unless you really know what you are doing.
\documentclass[letterpaper,english]{article}
\usepackage[T1]{fontenc}
\usepackage[latin9]{inputenc}
\usepackage{setspace}
\usepackage[authoryear]{natbib}
\doublespacing

\makeatletter

%%%%%%%%%%%%%%%%%%%%%%%%%%%%%% LyX specific LaTeX commands.
\special{papersize=\the\paperwidth,\the\paperheight}


\makeatother

\usepackage{babel}
\begin{document}

\section{Automatic sentiment indexes}

We analyse articles of the economics section of the weekly newspaper
\textit{The New York Times} obtained using $LexisNexis$, a research
tool, that have been published between 1990:06 and 2013:08 from its
publically available online archive. We extract the sentiment of news
articles using a lexicon based algorithm provided by \citet{rinker})
in the open Software project R (see \citealp{rlanguage}). 

Each article is attributed a sentiment value, $ArticleValue$, defined
as: 
\begin{equation}
ArticleValue=\frac{\Sigma_{j}^{N}wordvalue_{j}\times n_{j}}{N},
\end{equation}


where $wordvalue_{j}$ is the sentiment value attributed to word $j$,
where $j=1,...,N$ and $N$ is the number of words of the respective
article. The value is based on the lexicon provided by \citet{hu2004mining},
a publicly available English-language resource for sentiment analysis.
It contains 4776 negative and 2003 positive words. 

The articles are evaluated sentence by sentence. Each positive word
is counted as 1 and each negative word as -1. Then, the value found
in the lexicon is refined in the following way. If negators like ``$not$''
are found in the proximity (4 words before and 2 words after the word)
and if the number of negators is odd the polarity of a word is inverted.
Furthermore, if there are amplifiers like ``$very$'' or de-amplifiers
``$little$'' are in the proximity, the value is increased respectively
decreased. For a detailed description of the algorithm see \citet{rinker}.

\bibliographystyle{plainnat}
\bibliography{MTI,mti_ip}

\end{document}
